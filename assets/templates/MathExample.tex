\documentclass{article}
\usepackage{amsmath}

\begin{document}

    \section*{Multiple Examples of Mathematical Expressions}

    \subsection*{Example 1: Quadratic Formula}

    The solutions to the quadratic equation $ax^2 + bx + c = 0$ are given by:
    \begin{equation}
        x = \frac{-b \pm \sqrt{b^2 - 4ac}}{2a}
    \end{equation}

    \subsection*{Example 2: Summation}

    The sum of the first $n$ natural numbers is given by:
    \begin{equation}
        \sum_{i=1}^{n} i = \frac{n(n+1)}{2}
    \end{equation}

    \subsection*{Example 3: Pythagorean Theorem}

    In a right-angled triangle with sides $a$, $b$, and $c$ (where $c$ is the hypotenuse), the Pythagorean theorem holds:
    \begin{equation}
        a^2 + b^2 = c^2
    \end{equation}

    \subsection*{Example 4: Binomial Theorem}

    The expansion of $(a + b)^n$ using the binomial theorem is given by:
    \begin{equation}
    (a + b)^n = \sum_{k=0}^{n} \binom{n}{k} a^{n-k} b^k
    \end{equation}

    \subsection*{Example 5: Fourier Transform}

    The Fourier transform of a function $f(t)$ is defined as:
    \begin{equation}
        F(\omega) = \int_{-\infty}^{\infty} f(t) e^{-i\omega t} \,dt
    \end{equation}

    \subsection*{Example 6: Lorentz Transformation}

    The Lorentz transformation equations for time $t'$ and space $x'$ in special relativity are given by:
    \begin{align}
        t' & = \gamma \left(t - \frac{vx}{c^2}\right) \\
        x' & = \gamma (x - vt)
    \end{align}
    where $\gamma = \frac{1}{\sqrt{1 - \frac{v^2}{c^2}}}$.

\end{document}
