\documentclass{article}
\usepackage{graphicx}
\usepackage{amsmath}
\usepackage{lipsum}

\title{A Comprehensive Study on LaTeX}
\author{Your Name}
\date{\today}

\begin{document}

    \maketitle

    \section{Introduction}
    \label{sec:intro}
    \lipsum[1-2]

    \section{Methodology}
    \label{sec:method}
    \lipsum[3]

    \subsection{Data Collection}
    \label{subsec:data}
    We collected data from various sources.

    \subsection{Experimental Setup}
    \label{subsec:setup}
    The experimental setup included...

    \section{Results}
    \label{sec:results}
    The results are presented in Table \ref{tab:results}.

    \begin{table}[htb]
        \centering
        \begin{tabular}{|c|c|}
            \hline
            Method & Accuracy \\
            \hline
            Method A & 90\% \\
            Method B & 85\% \\
            \hline
        \end{tabular}
        \caption{Experimental Results}
        \label{tab:results}
    \end{table}

    \subsection{Analysis}
    \label{subsec:analysis}
    We analyze the results in Equation \eqref{eq:analysis}.

    \begin{equation}
        \label{eq:analysis}
        F(x) = \int_{a}^{b} f(x) \, dx
    \end{equation}

    \section{Discussion}
    \label{sec:discussion}
    The discussion includes a comparison with related work.

    \begin{figure}[htb]
        \centering
        \includegraphics[width=0.8\textwidth]{example-image}
        \caption{A Sample Figure}
        \label{fig:sample}
    \end{figure}

    \lipsum[4-5]

    \section{Conclusion}
    \label{sec:conclusion}
    In conclusion, LaTeX is a powerful typesetting system for producing professional documents.

    \section*{Acknowledgments}
    I would like to thank...

    \bibliographystyle{plain}
    \bibliography{references}

\end{document}
